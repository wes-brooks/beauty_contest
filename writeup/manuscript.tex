\documentclass[authoryear,review, 12pt]{elsarticle}



\newcommand{\maxwidth}{\textwidth}

\usepackage{alltt}
\usepackage[T1]{fontenc}
\usepackage{geometry}
\geometry{verbose}
\setlength{\parskip}{\bigskipamount}
\setlength{\parindent}{0pt}
\usepackage{bm}
\usepackage{amsthm}
\usepackage{amsmath}
\usepackage{amssymb}
\usepackage{undertilde}
\usepackage{graphicx}
\usepackage{graphics}
\usepackage{setspace}
\usepackage{esint}
\usepackage{booktabs}
\usepackage{color}
\usepackage{multirow}
\usepackage{natbib}
\usepackage{ifthen}
\usepackage{longtable}

\usepackage{setspace}
\doublespacing

\usepackage{lineno}
\linenumbers

\mathchardef\mhyphen="2D % Define a "math hyphen"

\newcommand{\hlc}[2][yellow]{ {\sethlcolor{#1} \hl{#2}} }
\newcommand{\highlight}[1]{\colorbox{yellow}{$\displaystyle #1$}}

\newtheorem{thm}{Theorem}
\newtheorem{lem}{Lemma}

\newcommand\pr{\mathbf P}
\newcommand{\E}{\mathbb{E}}

\setlength{\textwidth}{6in}
\setlength{\textheight}{8in}
\textwidth=6in
\textheight=8in
\setlength{\topmargin}{0in}
\setlength{\oddsidemargin}{0in}
\setlength{\evensidemargin}{0in}

\journal{}
\date{}

\newboolean{thesis}
\setboolean{thesis}{false}

\begin{document}


\begin{frontmatter}

\title{Predicting recreational water quality advisories: a comparison of statistical methods}

\author[usgs-wiwsc]{Wesley Brooks}
\ead{wrbrooks@usgs.gov}

\author[usgs-wiwsc]{Rebecca Carvin}
\ead{rbcarvin@usgs.gov}

\author[usgs-wiwsc]{Steven Corsi}
\ead{srcorsi@usgs.gov}

\author[usgs-wiwsc]{Michael Fienen}
\ead{mnfienen@usgs.gov}

\address[usgs-wiwsc]{Wisconsin Water Science Center, United States Geological Survey, 8505 Research Way, Middleton, WI 53562}

\begin{abstract}
Epidemiological studies have indicated that the concentration of fecal indicator bacteria (FIB) in beach water is associated with illnesses among people who have contact with the water. In order to mitigate public health impacts, many beaches are managed so that an advisory is posted when the concentration of FIB exeeds a beach action value. The most commonly used method of measuring FIB concentration takes $18 - 24$ hours before returning a result. It has become common to base beach management decisions on the output from nowcast models that use environmental and meteorological conditions to predict the current concentration of FIB, avoiding the 24h lag. Most commonly, nowcast models are estimated using ordinary least squares regression, but other regression methods from the statistical and machine learning literature are appearing in growing numbers. The choice of regression method is quite important to the accuracy of the nowcast model, and the literature comparing the performance of different methods has often made those comparisons at a single site, which may or may not be representative. We compare several regression methods to identify which produces the most accurate predictions. The comparison is made at several sites in Wisconsin, including beaches on Lake Superior and Lake Michigan. A random forest model is identified as the most accurate. That is followed by the adaptive Lasso, which also includes a variable selection step that reduces the number of covariates that need to be measured in order to make predictions.
\end{abstract}

\begin{keyword}
Beach water quality, Statistical model, Performance evaluation, Real-time prediction
\end{keyword}

\end{frontmatter}

\section{Introduction}\label{introduction}

Fecal indicator bacteria (FIB) in beach water are often used to indicate
contamination by harmful pathogens \citep{Cabelli:1979lb,Wade:2006qc,Wade:2008yi,Fleisher:2010xo}. The United States Environmental
Protection Agency (USEPA) has established, through epidemiological
studies, that FIB concentration is associated with human health outcomes
\citep{Cabelli:1983od,Dufour:1984yn,USEPA:ecs}. Accordingly, many
states have established regulatory standards for water quality;
Wisconsin states that a beach should be posted with a swimmer's advisory
when the concentration of the FIB \emph{Escherichia coli} exceeds the
beach action value (BAV) of \(235\) colony forming units (CFU) / \(100\)
mL \citep{USEPA-2012,WDNR-2012}. The BAV of \(235\) CFU / \(100\) mL
was recommended by the USEPA as the ``do not exceed'' threshold in order
to limit gastrointestinal illnesses among those coming into contact with
beach water to 36 cases per 1000 people \citep{USEPA-2012}. Traditional
analysis methods for FIB concentration requires \(18-24\) hours for
culturing a sample, so the decision to post an advisory is often made
based on the previous day's FIB concentration, which is the so-called
``persistence model'' for beach management \citep{USEPA:2007lj}. Previous
research has shown that the concentration of FIB in beach water can vary
substantially during the \(18-24\) h analysis period, with the result
that the persistence model often provides incorrect information for
posting warnings \citep{Whitman:2004wv,Whitman:2008nb}. Thus, at
beaches managed using the persistence model, the public is sometimes
exposed to health risks or unnecessarily deprived of recreation
opportunities.

In order to have more immediate knowledge of the FIB concentration, it
is now common to use regression models that ``nowcast'' the FIB
concentration based on easily observed surrogate covariates,
e.g.~turbidity and running $24$ h rainfall total \citep{Brandt:2006gj,Olyphant:2004yq}. Numerous regression techniques have been used to
generate nowcast models of FIB concentration. The techniques include
ordinary least squares (OLS) \citep{Nevers:2005ln,Francy:2007yv},
partial least squares (PLS) \citep{Hou:2006nf,Brooks-Fienen-Corsi-2013}, logistic regression \citep{Waschbusch:2004bd,Jin:2006tr}, decision trees \citep{Stidson-2012}, random forests
\citep{Parkhurst:2005zf,Jones-Liu-Dorovitch-2012}, and artificial
neural networks \citep{Kashefipour-Lin-Falconer-2005,He:2008jx}. A
thorough review of the regression techniques being used in nowcast
models for FIB concentration is provided by
\citep{deBrauwere-Koffi-Servais-2014}. An assessment of seven methods of
regression for FIB concentration in beach water at Santa Monica Beach in
California identified classification trees, artificial neural networks,
and logistic regression as the three best methods
\citep{Thoe-Gold-Griesbach-Grimmer-Taggart-Boehm-2014}.

Ordinary least squares (OLS) regression is the most commonly used
regression technique in the nowcast models
\citep{deBrauwere-Koffi-Servais-2014}. However, OLS is well-known for
drawbacks like overfitting, difficulty of covariate selection, and the
inflexibility of its linear modeling structure \citep{Ge:2007ou}. The
literature suggests that many regression techniques have been
successfully used for nowcast modeling, but due to differences in such
factors as local conditions, data handling, and performance validation,
it is not possible to identify the best regression technique for nowcast
modeling by comparing different models at different sites. In this
study, fourteen regression techniques are evaluated in nowcast models at
seven Wisconsin Great Lakes beaches with four years of data. The results
are compared to identify the techniques that most accurately predict
instances when a swimmer's advisory should be posted. This comparison is
designed to provide insights that may be lost when comparing individual
methods at single sites.

The remainder of the paper is organized as follows: in the next section
we discuss data collection and handling, describe the regression
techniques, and explain how the comparisons were made. Next, we present
the results of comparing the methods by several metrics including: area
under the receiver operating characteristic (ROC) curve; predictive
error sum of squares; and raw number of correct/incorrect predictions.
Finally, we discuss what the comparison suggests about which are the
best choices for a regression technique in a nowcast model.

\section{Data}\label{data}

The seven beach sites analyzed in this study are located within two
distinct regions of Wisconsin (Figure 1). Three of the
sites are on Chequamegon Bay in Lake Superior and the remaining five are
in Manitowoc County in Lake Michigan. For each site in the study, the
data used to estimate the predictive models for FIB concentration were
measured by a combination of automatic sensing, hydrodynamic and
atmospheric modeling, and manual sampling. A listing of the covariates
included for modeling the FIB concentration at each beach site is in the
Appendix.

\subsection{Site descriptions}\label{site-descriptions}

\subsubsection{Chequamegon Bay sites}\label{chequamegon-bay-sites}

Chequamegon Bay is approximately \(19\) km long and ranges from 3 to 10
km in width, with a maximum depth of \(11\) m. Water quality at the
three Chequamegon Bay/Lake Superior beaches is influenced by nearby
streams, as well as by urban runoff from Ashland and Washburn,
Wisconsin. Thompson beach is within the small town of Washburn, on the
north side of the bay. Next to the beach are a playground, RV campsites,
piers and boat launch. There are two flowing artesian wells that drain
to the beach. Thompson Creek, about \(300\) m from the beach, is the
nearest stream. Maslowski beach is on the west side of Ashland, on the
south side of the Chequamegon Bay. A playground and parking area are
near the beach. Two flowing artesian wells are near the swim area, and
Fish Creek (\(1.5\) km west of the beach) and Whittlesley Creek (\(3\)
km northwest) are the nearest streams. Kreher beach is in Ashland, \(4\)
km northeast of Maslowski beach. Kreher Park has an RV campground,
playground and boat launch, and is nearest to Bay City Creek, which is
\(1\) km east of the beach. All of the listed streams are influenced by
areas of agricultural and forested land use, with Bay City Creek also
influenced by urban land use \citep{Francy-et-al-2013}. Contributions
from Fish Creek are dynamic due to a wetland at the creek's outlet that
is influenced by the lake level.

\subsubsection{Manitowoc County sites}\label{manitowoc-county-sites}

Red Arrow beach is within the city of Manitowoc. It has numerous
potential influences on water quality, including the mouth of the
Manitowoc River one mile north and urban runoff draining to the beach
through storm sewer outlets. The Manitowoc River is dominated by
agricultural land use, but there is some urban influence from the city
of Manitowoc. The Manitowoc sewage treatment plant sits at the mouth of
the Manitowoc River. Neshotah beach is in the small community of Two
Rivers. Small storm sewers drain to the north and to the south directly
adjacent to the beach boundaries, and the mouth of the Twin River is
\(1\) km south of the beach. The Twin River drains an agricultural
watershed. Point Beach State Park is approximately \(18\) km north of
Manitowoc, about \(4\) km north of the mouth of Molash Creek whose
watershed encompasses a mix of agricultural land use and wetland area.
The mouth of Twin River is \(10\) km south and the mouth of the Kewaunee
River is \(26\) km north of Point Beach State Park. The Kewaunee River
is also dominated by agricultural land use. FIB concentration was
measured at three beaches within Point Beach State Park, with the
samples being considered in the models as three independent
observations. Hika beach is south of the city of Manitowoc near the
small community of Cleveland. Large floating mats of \emph{Cladophora}
algae are common. Centerville Creek, a small stream dominated by
agricultural land use, drains to the lake adjacent to the beach.

\subsection{Data sources}\label{data-sources}

Field data collection and sample analysis followed methods described in
\cite{Francy-et-al-2013}. Concentration of \emph{E. coli} was measured at each
beach \(2-4\) times per week for \(12-14\) weeks between Memorial Day
and Labor Day from 2010 through 2013. Samples were collected from the
center of the length of the beach, \(30\) cm below the water surface
where total water depth was \(60\) cm. All samples were quantified by
use of the Colilert® QuantiTray/2000 method, which were reported as the
most probable number (MPN) of \emph{E. coli} colony forming units (CFU)
and were read after \(24\) hours of incubation \citep{Colilert}.

Additional covariates were compiled from a variety of sources including
online data and manual measurements. Online data were accessed using
Environmental Data Discovery and Transformation (EnDDaT), a web service
that accesses data from a variety of sources, compiles and processes the
data, and performs common transformations \citep{EnDDaT-2014}. Three
sources of data were accessed: The U.S. Geological survey National Water
Information System (NWIS) \citep{NWIS}, the National Weather Service
North Central River Forecasting Center \citep{NCRFC}, and the Great Lakes
Costal Forecasting System \citep{Schwab-Bedford-1999}. Covariates
acquired through these sources included: river discharge, precipitation,
lake current vectors, wave height, wave direction, lake level, water
temperature, air temperature, wind vector, and percent cloud cover.

Most covariates from online sources were available in hourly increments
with the exception of NWIS data which were available in 15 minute
increments. In order to make best use of this high-frequency data for
daily predictions, several summary statistics were calculated over
several time windows for use as potential covariates. The use of $1$, $2$,
$6$, $12$, $24$, $48$, $72$, and $120$ hour time windows for calculating the summary
statistics followed recent research showing that selecting from windowed
and lagged versions of raw high-frequency covariates can improve the
predictive accuracy of regression models
\citep{Cyterski-Zhang-White-Molina-Wolfe-Parmar-Zepp-2012}. The choice of
summary statistics to include as potential covariates was guided by
scientific judgement regarding phenomena that could affect the FIB
concentration. For example, standard deviation of water temperature
measurements over the window period reflected the variability in water
temperature, which may affect the survival and growth of FIB; the sum of
rainfall measurements over the window period indicated the magnitude of
recent rain events, which may be associated with FIB washed into the
lake from sources on land; and the mean of cloud cover measurements over
the window period may measure the degree to which UV light was inhibited
from breaking down FIB colonies in the water. The summary statistics
computed by EnDDaT were the mean, minimum, maximum, difference, sum, and
standard deviation.

Manually observed data were instantaneous observations that had the
benefit of being measured when and where the FIB samples were collected.
However, these covariates were measured only once per day and at greater
expense than the online data because the data had to be collected by
field personnel. Manual data collection techniques were guided by the
USEPA's Great Lakes Beach Sanitary Survey \citep{USEPA:2008ib}. Among the
manually measured data were turbidity, wave height, number of birds
present, number of people present, amount of algae floating in the swim
area and on the beach, specific conductance, water and air temperature,
wind direction, and wind speed. Every beach dataset included turbidity,
but other field covariates occasionally had to be dropped from some of
the datasets because of missing values or questionable reliability.

\subsection{Data transformations}\label{data-transformations}

The response for the continuous regression models was the base-10
logarithm of the FIB concentration. For the binary regression models,
the response is an indicator of whether the concentration exceeds the
BAV. Transformations were applied to some of the covariates during
pre-processing: the beach water turbidity and the discharge of
tributaries near each beach were log-transformed, and rainfall
covariates were all square root transformed. These transformations were
based on the performance of previous studies and were applied to all
datasets \citep{Ge:2007ou,Frick:2008jo}.

\section{Methods}\label{methods}

\subsection{Definitions}\label{definitions}

For each site, let \(\bm{y}=(y_1, \dots, y_n)\) be the vector of
\(\log_{10}\) FIB concentration measurements, let \(n\) be the number of
observations, and let \(p\) be the number of explanatory covariates. The
beach action value (BAV) of \(235\) CFU / \(100\) mL is represented
symbolically in equations by \(\delta\). Define an exceedance as a
meansured FIB concentration that exceeds the BAV. Conversely, a
nonexceedance is a measured FIB concentration that does not exceed the
BAV.

Predictions are the result of applying a model to data that was not used
to estimate the model. The predicted \(\log_{10}\) FIB concentration is
denoted by a tilde (e.g., \(\tilde{y}_i\)). On the other hand, applying
the model to the same data as was used to estimate the model produces
fitted values, which are denoted by a hat (e.g., \(\hat{y}_j\)). We
define a predicted exceedance as when a model predicts that the FIB
concentration exceeds the BAV. This is not the same as
\(\tilde{y}_i > \delta\) because predictions are compared to a decision
threshold \(\hat{\delta}\) rather than to the BAV \(\delta\). The
decision threshold \(\hat{\delta}\) is a parameter that can be adjusted
to tune the predictive performance. For instance, increasing the
decision threshold reduces the number of false positives but increases
the number of false negatives. Setting the decision threshold is an
important detail that is discussed in Section 5.4.

\subsection{Listing of statistical
techniques}\label{listing-of-statistical-techniques}

Fourteen different regression modeling techniques were considered (Table
1). Each technique uses one of five modeling algorithms: the gradient
boosting machine (GBM), the adaptive Lasso (AL), the genetic algorithm
(GA), partial least squares (PLS), or sparse PLS (SPLS). Each technique
is applied to either continuous or binary regression and to either
covariate selection and model estimation, or covariate selection only.

\subsubsection{Continuous vs.~binary
regression}\label{continuous-vs.binary-regression}

The goal of predicting exceednaces of the water quality standard is
approached in two ways: one is to predict the bacterial concentration
and then compare the prediction to a threshold, which is referred to as
continuous modeling. The other is referred to as binary modeling, in
which we predict the state of the binary indicator \(z_{i}\):

\[ z_{i}=\left\{ \begin{array}{cl}
0 & \text{ if } y_{i} < \delta\\
1 & \text{ otherwise}
\end{array}\right. \]

where \(y_i\) is the FIB concentration and \(\delta\) is the BAV. The
indicator is coded as zero when the concetration is below the BAV and
one when the concentration exceeds the BAV. All of the binary modeling
techniques herein use logistic regression \citep{Hosmer-Lemeshow-2004}.
Binary regression methods are indicated with a (b).

\subsubsection{Weighting of observations in binary
regression}\label{weighting-of-observations-in-binary-regression}

The concentration of FIB in the water at a single beach on a single day
can be subject to a large degree of spatiotemporal heterogeneity
\citep{Whitman:2004pc}. Thus, when the concentration in a sample is
observed to fall near the BAV, there is considerable uncertainty as to
whether an independent sample from the same date and location would or
would not exceed the BAV. A weighting scheme for the binary regression
techniques was designed to reflect this ambiguity by giving more weight
to observations far from the BAV. In the weighting scheme, observations
were given weights \(w_i\) for \(i=1,\dots,n\), where

\[
    \begin{aligned}
        w_i &= (y_i - \delta) / \hat{\rm{sd}}(y)\\
        \hat{\rm{sd}}(y) &= \sqrt{\sum_{i=1}^n (y_i - \bar{y})^2 / n}\\
        \bar{y} &= \sum_{i=1}^n y_i / n.
    \end{aligned}
\]

That is, the weights are equal to the number of standard deviations that
the observed concentration lies from the BAV. All the techniques that
were implemented with this weighting scheme were separately implemented
without any weighting of the observations. The methods using the
weighting scheme are indicated by (w).

\subsubsection{Selection-only methods}\label{selection-only-methods}

Modeling methods indicated by an (s) were applied only to select
covariates for a regression model. Once the covariates were selected,
the regression model using the selected covariates was estimated using
ordinary least squares for the continuous methods, or ordinary logistic
regression for the binary methods.

\subsubsection{Listing of modeling
algorithms}\label{listing-of-modeling-algorithms}

\paragraph{GBM}\label{gbm}

A GBM model is a so-called random forest model - a collection of many
regression trees, each fit to a randomly drawn subsample of the training
data \citep{Friedman-2001}. Prediction is done by averaging the outputs
of the trees. Two GBM-based techniques were studied - we refer to them
as GBM-OOB and GBM-CV. The difference between these two techniques is in
how the optimal number of trees is determined - GBM-CV selects the
number of trees in a model using leave-one-out cross validation (CV),
while GBM-OOB uses the so-called out-of-bag error estimate, where the
predictive error of each tree is estimated by its predictive error over
the observations that were left out when fitting the tree. In contrast,
the predictive error of CV is estimated from observations that are left
out from the training data altogether, and are therefore not used in the
fitting of any trees. The CV method is much slower (it has to construct
as many random forests as there are observations, while the OOB method
only requires computing a single random forest). However, GBM-CV should
more accurately estimate the prediction error.

\paragraph{Adaptive Lasso}\label{adaptive-lasso}

The least absolute shrinkage and selection operator (Lasso) is a
penalized regression method that simultaneously selects relevant
covariates and estimates their coefficients \citep{Tibshirani-1996}. The
AL is a refinement of the Lasso that possesses the so-called ``oracle''
properties of asymptotically selecting exactly the correct covariates
and estimating them as accurately as would be possible if their
identities were known in advance \citep{Zou-2006}. To use the AL for
prediction requires selecting a tuning parameter. In this study, the AL
tuning parameter \(\lambda\) was selected to minimize the corrected
Akaike Information Criterion (\(\text{AIC}_{\tt c}\)) \citep{Akaike-1973,Hurvich-Simonoff-Tsai-1998}. The \(\text{AIC}_{\tt c}\) for the
continuous regression models is

\begin{align}\label{eq:AICc-continuous}
\text{AIC}_{\tt c} = \sum_{i=1}^n \frac{(y_i - \hat{y}_i)^2}{\hat{\sigma}^2} + 2df + \frac{2df(df+1)}{n-df-1}.
\end{align}

where \(\hat{\sigma}^2\) is the variance estimate from the model that
used all of the covariates, \(n\) is the sample size, \(y_i\) and
\(\hat{y}_i\) are respectively the observed and fitted value of the
\(i\)th FIB measurement, and \(df\) is the number of covariates in the
model. For binary regression models, the \(\text{AIC}_{\tt c}\) is

\begin{align}\label{eq:AICc-binary}
\text{AIC}_{\tt c} = \sum_{i=1}^n  2\left\{ z_i \log (\frac{z_i}{\hat{z}_i}) + (1-z_i) \log \left( \frac{1-z_i}{1-\hat{z}_i} \right) \right\} + 2df + \frac{2df(df+1)}{n-df-1}
\end{align}

where \(z_i\) and \(\hat{z}_i\) are respectively the observed and fitted
value of the \(i\)th BAV exceedance indicator.

\paragraph{Genetic algorithm}\label{genetic-algorithm}

The GA was used to select covariates for either an OLS or a logistic
regression model. By analogy to natural selection, so-called chromosomes
in the GA represent regression models \citep{Fogel-1998}. A covariate is
included in the model if the corresponding element of the chromosome is
one, but not otherwise. Chromosomes are produced in successive
generations, where the first generation is produced randomly and
subsequent generations are produced by combining chromosomes from the
current generation, with additional random drift. The chance that a
chromosome in the current generation produces offspring in the next
generation is an increasing function of its fitness. The fitness of each
chromosome is calculated by the \(\text{AIC}_{\tt c}\)
(\ref{eq:AICc-continuous}), (\ref{eq:AICc-binary}).

\paragraph{PLS}\label{pls}

Partial least squares (PLS) regression is a tool for building regression
models with many covariates \citep{Wold-Sjostrum-Eriksson-2001}. PLS
works by decomposing the covariates into mutually orthogonal components,
with the components then used as the covariates in a regression model.
This is similar to principal components regression (PCR), but the way
PLS components are chosen ensures that they are aligned with the
response, whereas PCR is sometimes criticised for decomposing the
covariates into components that are unrelated to the response. To use
PLS, one must decide how many components to use in the model. Following
\citep{Brooks-Fienen-Corsi-2013}, we use the PRESS statistic to select
the number of components.

\paragraph{SPLS}\label{spls}

Sparse PLS (SPLS) combines the orthogonal decompositions of PLS with the
sparsity of Lasso-type covariate selection \citep{Chun-Keles-2010}. To do
so, SPLS uses two tuning parameters: one that controls the number of
orthogonal components and one that controls the Lasso-type penalty. The
optimal parameters are those that minimize the mean squared prediction
error (MSEP) over a two-dimensional grid search. The MSEP is estimated
by 10-fold cross-validation.

\begin{table}
\centering
\begin{tabular}{lcccc}                       
Name & Algorithm & Binary & Weighted & \begin{tabular}{c}Selection\\Only\end{tabular} \\
\hline
GBM-OOB & Gradient boosting & & & \\
GBM-CV & Gradient boosting     & & & \\
AL & Adaptive Lasso & & & \\
AL (s) & Adaptive Lasso & & & X \\
AL (b) & Adaptive Lasso & X & & \\
AL (b,w)  &  Adaptive Lasso & X & X & \\
AL (s,b) & Adaptive Lasso & X & & X \\
AL (s,b,w) & Adaptive Lasso & X & X & X \\
GA & Genetic algorithm & & & \\
GA (b) & Genetic algorithm & X & & \\
GA (b,w) & Genetic algorithm & X & X & \\
PLS & Patrial least squares & & & \\
SPLS & Sparse partial least squares & & & \\
SPLS (s) & Sparse partial least squares & & & X \\
\end{tabular}
\caption{Comprehensive list of the modeling methods analyzed in this study. Listed for each method are the method's abbreviation, the algorithm used by the method, and indicators of whether the method uses binary regression, observation weighting, and/or covariate selection separately from estimation. The two methods GBM-CV and GBM-OOB differ in how the number of GBM trees are selected, as described in the main text.}
\label{table:methods}
\end{table}



\subsection{Cross Validation}\label{cross-validation}

Our assessment of the modeling techniques was based on their performance
in predicting exceedances of the BAV. Two types of cross validation were
used to measure the performance in prediction: leave-one-out (LOO) and
leave-one-year-out (LOYO). In LOO CV, one day's observation was held out
for validation while the rest of the data was used to train a model. At
Point Beach State Park, where FIB concentration was measured at three
locations each day, all three daily observations were left out of the
LOO CV models together. The model was used to predict the result of the
held out observation(s), and the process - including estimating a new
predictive model - was repeated for each date with available data. On
the other hand, each cycle of LOYO CV held out an entire year's worth of
data for validation instead of a single observation. It was intended to
approximate the performance of the modeling technique under a typical
use case: a new model is estimated before the start of each annual beach
season and then used for predicting exceedances during the season. The
LOYO models in this study were estimated using all the available data
except for the held out year, even that from future years. So for
instance the 2012 models were estimated using the 2010-2011 and 2013
data.

Some methods also used CV internally to select tuning parameters. In
those cases the internal CV was conducted by subdividing the model data,
and never looking at the held-out observation(s). This process was
independent of the CV to assess predictive performance.

\subsection{Comparing methods, and quantifying uncertainty in the
ranks}\label{comparing-methods-and-quantifying-uncertainty-in-the-ranks}

Results were compiled into one table for each site where each
observation corresponds to a row in the table. For example, a few rows
from the results table at Hika are presented in Table 2. The results
table has a column for the observed \(\log_{10}\) FIB concentration and,
for each method, columns for the predicted concentration by LOO CV and
by LOYO CV. From the table, performance of the modeling methods was
summarized by calculating the area under the receiver operating
characteristic (ROC) curve (AUROC) and the predictive error sum of
squares (PRESS).

\begin{table}
\centering
\begin{tabular}{ccccccc}
Row & \begin{tabular}{c}$\log_{10}$\\FIB\end{tabular} & \begin{tabular}{c}PLS\\(LOO)\end{tabular} & \begin{tabular}{c}PLS\\(LOYO)\end{tabular} & $\cdots$ & \begin{tabular}{c}SPLS\\(LOO)\end{tabular} & \begin{tabular}{c}SPLS\\(LOYO)\end{tabular} \\
\hline
1 & 2.54 & 2.35 & 2.22 & \dots & 2.29 & 2.55 \\
2 & 2.59 & 1.87 & 1.79 & \dots & 1.91 & 1.23 \\
\vdots & \vdots & \vdots & \vdots & \dots & \vdots & \vdots \\
166 & 1.57 & 1.93 & 2.06 & \dots & 1.83 & 2.07 \\
167 & 3.38 & 1.84 & 2.01 & \dots & 1.80 & 1.71
\end{tabular}
\caption{An example of how the results
for a site (Hika here) were compiled into a results table. The summary
statistics used to compare predictive performance (area under the ROC
curve and predictive error sum of squares) were calculated from the
table. Confidence intervals for the summary statistics were computed via
the bootstrap by resampling (with replacement) the rows of the results
table.}
\label{table:hika-results}
\end{table}

The ROC curve is an assessment of how well predictions are separated
into exceedances and nonexceedances \citep{Hanley-McNeil-1982}. Every
possible value of the decision threshold \(\hat{\delta}\) corresponds to
a point on the ROC curve, with coordinates
\((1-\text{specificity}(\hat{\delta}), \text{sensitivity}(\hat{\delta}))\).
Specificity is the fraction of decision threshold non-exceedances that
have been correctly predicted.Sensitivity is the fraction of decision
threshold exceedances that have been correctly predicted. Specificity
and sensitivity are mathematically defined as

\begin{align*}
\text{specificity}(\hat{\delta}) &= \sum_{i=1}^n I(\tilde{y}_i \le \hat{\delta}) I(y_i \le \delta) / \sum_{j=1}^n I(y_j \le \delta) \\
\text{sensitivity}(\hat{\delta}) &= \sum_{i=1}^n I(\tilde{y}_i > \hat{\delta}) I(y_i > \delta) / \sum_{j=1}^n I(y_j > \delta).
\end{align*}

where \(I(A)\) is the indicator function that takes value one if \(A\)
is true and zero is \(A\) is false.

The AUROC averages the model's performance over the range of possible
thresholds. A model which perfectly separates exceedances from
non-exceedances in prediction would have an AUROC of one, while a model
that predicts exceedances no better than a coin flip would have an
expected AUROC of \(0.5\).

While AUROC quantifies how well a model sorts exceedances and
non-exceedances, PRESS measures how accurately a model's predictions
match the observed FIB concentration. The PRESS can only be computed for
continuous regression methods. Recalling that the \(i\)th observed
\(\log_{10}\) FIB concentration is denoted \(y_{i}\) and that the
corresponding prediction is denoted \(\tilde{y}_i\) for \(i=1,\dots,n\)
where \(n\) is the total number of predictions, the PRESS is given by

\[\text{PRESS}=\sum_{i=1}^{n}\left(\tilde{y}_{i}-y_{i}\right)^{2}.\]

To identify which modeling methods had the best performance across all
sites, the methods at each site were ranked from worst to best according
to AUROC and PRESS (the ranks were taken worst to best so that larger
numbers represent better performance). The mean rank of each method was
then taken across the sites as a measurement of how each of our modeling
methods performed relative to the others. Uncertainty in the rankings
was quantified by the bootstrap: since PRESS and AUROC are functions of
the results tables, the bootstrap procedure was carried out by
resampling the rows of each results table and recalculating the ranks
for each bootstrap sample. We used \(1,000\) bootstrap samples of each
results table in the analysis that follows.

\section{Results}\label{results}

\subsection{AUROC}\label{auroc}

The mean LOO and LOYO ranks were computed for all of the methods as
determined by AUROC (Figure 2). The three top-ranked methods were
GBM-CV, GBM-OOB, and AL. In order to facilitate a pairwise comparison
between modeling methods, the frequency that the mean AUROC rank of
GBM-OOB, GBM-CV, or AL exceeded each of the other modeling methods for
the leave-one-year-out and the leave-one-out analyses were also computed
(Table 3).

\begin{table}
    \resizebox{\textwidth}{!}{
    \begin{tabular}{rccccccccccccc}

\multicolumn{6}{l}{\kern-2em \textbf{Leave-one-year-out cross-validation:}} & & & & & & & & \\
 & \begin{tabular}{c}GBM- \\ OOB\end{tabular} & \begin{tabular}{c}AL\end{tabular} & \begin{tabular}{c}AL \\ (s)\end{tabular} & \begin{tabular}{c}AL \\ (b,w)\end{tabular} & \begin{tabular}{c}SPLS\end{tabular} & \begin{tabular}{c}PLS\end{tabular} & \begin{tabular}{c}AL \\ (b)\end{tabular} & \begin{tabular}{c}SPLS \\ (s)\end{tabular} & \begin{tabular}{c}AL \\ (b,s)\end{tabular} & \begin{tabular}{c}AL \\ (b,w,s)\end{tabular} & \begin{tabular}{c}GA\end{tabular} & \begin{tabular}{c}GA \\ (b,w)\end{tabular} & \begin{tabular}{c}GA \\ (b)\end{tabular} \\ 
  \hline
GBM-CV & 0.86 & 0.87 & 0.98 & 0.99 & 1.00 & 1.00 & 1.00 & 1.00 & 1.00 & 1.00 & 1.00 & 1.00 & 1.00 \\ 
  GBM-OOB &  & 0.69 & 0.92 & 0.95 & 1.00 & 0.99 & 0.99 & 1.00 & 1.00 & 1.00 & 1.00 & 1.00 & 1.00 \\ 
  AL &  &  & 0.96 & 0.87 & 1.00 & 0.97 & 0.98 & 1.00 & 1.00 & 1.00 & 1.00 & 1.00 & 1.00 \\ 
  & & & & & & & & & & & & & \\
\multicolumn{6}{l}{\kern-2em \textbf{Leave-one-out cross-validation:}} & & & & & & & & \\
 & \begin{tabular}{c}GBM- \\ OOB\end{tabular} & \begin{tabular}{c}AL\end{tabular} & \begin{tabular}{c}AL \\ (b,w)\end{tabular} & \begin{tabular}{c}AL \\ (s)\end{tabular} & \begin{tabular}{c}AL \\ (b,w,s)\end{tabular} & \begin{tabular}{c}GA\end{tabular} & \begin{tabular}{c}SPLS\end{tabular} & \begin{tabular}{c}SPLS \\ (s)\end{tabular} & \begin{tabular}{c}PLS\end{tabular} & \begin{tabular}{c}GA \\ (b)\end{tabular} & \begin{tabular}{c}AL \\ (b)\end{tabular} & \begin{tabular}{c}AL \\ (b,s)\end{tabular} & \begin{tabular}{c}GA \\ (b,w)\end{tabular} \\ 
  \hline
GBM-CV & 0.96 & 1.00 & 1.00 & 1.00 & 1.00 & 1.00 & 1.00 & 1.00 & 1.00 & 1.00 & 1.00 & 1.00 & 1.00 \\ 
  GBM-OOB &  & 1.00 & 1.00 & 1.00 & 1.00 & 1.00 & 1.00 & 1.00 & 1.00 & 1.00 & 1.00 & 1.00 & 1.00 \\ 
  AL &  &  & 0.72 & 0.99 & 0.93 & 0.96 & 0.98 & 1.00 & 0.99 & 1.00 & 1.00 & 1.00 & 1.00 \\ 
  

    \end{tabular}}
    \caption{Under leave-one-year-out (top) and leave-one-out (bottom) cross validation, frequency of the mean AUROC rank of GBM-OOB, GBM-CV, or AL (in the rows) exceeding that of the other methods (in the columns).
    \label{tab:AUROC}}
\end{table}

\subsection{PRESS}\label{press}

The PRESS statistic is of interest because a good model should
accurately predict the bacterial concentration, but for assessing
regression models for FIB concentration, AUROC is more important than
PRESS because it directly measures the models' abilities to distinguish
exceedances from nonexceedances. That said, we expect the two statistics
to usually agree on which modeling methods perform best.

The top three techniques under both LOO and LOYO analysis as determined
by PRESS were GBM-CV, GBM-OOB, and AL (Figure 3). Again, to facilitate a
pairwise comparison between modeling methods, the frequency that the
mean PRESS rank of GBM-OOB, GBM-CV, or AL exceeded each of the other
modeling methods for the leave-one-year-out and the leave-one-out
analyses were computed (Table 4).

\begin{table}
    \centering
    \begin{tabular}{rccccccc}
    

\multicolumn{6}{l}{\kern-4em \textbf{Leave-one-year-out cross-validation:}} & & \\
 & \begin{tabular}{c}GBM- \\ OOB\end{tabular} & \begin{tabular}{c}AL\end{tabular} & \begin{tabular}{c}SPLS\end{tabular} & \begin{tabular}{c}PLS\end{tabular} & \begin{tabular}{c}SPLS \\ (s)\end{tabular} & \begin{tabular}{c}AL \\ (s)\end{tabular} & \begin{tabular}{c}GA\end{tabular} \\ 
  \hline
GBM-CV & 0.58 & 0.96 & 0.97 & 1.00 & 1.00 & 1.00 & 1.00 \\ 
  GBM-OOB &  & 0.94 & 0.96 & 1.00 & 1.00 & 1.00 & 1.00 \\ 
  AL &  &  & 0.52 & 0.88 & 0.94 & 0.99 & 1.00 \\ 
  & & & & & & & \\
\multicolumn{6}{l}{\kern-4em \textbf{Leave-one-out cross-validation:}} & & \\
 & \begin{tabular}{c}GBM- \\ OOB\end{tabular} & \begin{tabular}{c}AL\end{tabular} & \begin{tabular}{c}SPLS \\ (s)\end{tabular} & \begin{tabular}{c}SPLS\end{tabular} & \begin{tabular}{c}PLS\end{tabular} & \begin{tabular}{c}AL \\ (s)\end{tabular} & \begin{tabular}{c}GA\end{tabular} \\ 
  \hline
GBM-CV & 0.66 & 0.99 & 1.00 & 1.00 & 1.00 & 1.00 & 1.00 \\ 
  GBM-OOB &  & 0.97 & 1.00 & 1.00 & 1.00 & 1.00 & 1.00 \\ 
  AL &  &  & 0.73 & 0.80 & 0.85 & 0.93 & 1.00 \\ 
  
    \end{tabular}
    \caption{Under leave-one-year-out (top) or leave-one-out (bottom) cross validation, frequency of the mean PRESS rank of GBM-CV, GBM-OOB, or AL (in the rows) exceeding that of the other methods (in the columns).}
    \label{table:press.pairs.annual}
\end{table}

\subsection{Narrowing the focus}\label{narrowing-the-focus}

By AUROC and PRESS, and for LOO and LOYO analyses, the three
highest-ranked modeling methods were GBM-CV, GBM-OOB, and AL. The
fourth-ranked method was not consistent across the different analyses.
By the LOO CV analysis, AL was ranked better than the fourth-ranked
method by AUROC, AL (b,w), on 72\% of bootstraps and better than the
fourth-ranked method by PRESS, SPLS (s), on 73\% of bootstraps. And by
the LOYO CV analysis, AL was ranked better than the fourth-ranked method
by AUROC, AL (s), on 96\% of bootstraps and better than the
fourth-ranked method by PRESS, SPLS, on 52\% of bootstraps. Therefore,
we consider only the GBM methods and AL for the following analyses
because they consistently outperform the other methods.

\subsection{Classification of responses, and the decision
threshold}\label{threshold}

In operational use, a model's performance will be judged by how well it
distinguishes between exceedances and nonexceedances. While AUROC
measures how well exceedances and nonexcedances are sorted among the
predictions, AUROC is an average accuracy over all possible thresholds.
In order to provide the assessment most relevant for operational use,
the LOYO CV results were used to simulate how many correct and incorrect
predictions would be seen from an AL, GBM-OOB, or GBM-CV model with a
specific choice of decision threshold. Using the LOYO CV results
simulates the common scenario that a model is estimated at the beginning
of each beach season and used to make predictions during that season,
with a new model incorporating the new season of data estimated the
following year into the new model's training data.

Intuitively, the decision threshold should adapt to the conditions that
are observed in the beach's training data. If, for instance, exceedances
were rare in the training data, then we expect few exceedances in the
future, and should set the decision threshold high to reflect this
expectation. On the other hand, if the bacterial concentration often
exceeds the BAV, then the decision threshold should be set lower in
order to properly flag more of those exceedances. This intuition was
encoded into how the decision threshold was set for the LOYO models.
Specifically, the decision threshold \(\hat{\delta}\) was set to the
\(q^{th}\) quantile of the fitted values of non-exceedances in the
training set, where \(q\) is the proportion of training set observations
that are non-exceedances.

In Figure 4, we examine the counts on a per-beach basis of four
categories of decisions: true negatives (correct predictions of
nonexceedances), false positives (incorrect predictions of exceedances)
true positives (correct predictions of exceedances), and false negatives
(incorrect predictions of nonexceedances). In most cases, the counts
were similar between the three techniques, with GBM-OOB and GBM-CV
commonly resulting in a few more correct decisions than AL. There are,
however, exceptions where AL results had more correct decisions (e.g.,
Hika and Red Arrow).

\subsection{Covariate selection}\label{covariate-selection}

It was noted in Section 4.3 that GBM-OOB and AL are two of the three
best-ranked methods. One difference between the two is that AL does
covariate selection while GBM-OOB uses all of the available covariates.
We explore here how many covariates were used in AL models compared to
the GBM-OOB models (Figure 5).

At most of the sites, AL uses only a small fraction of the available
covariates, but at Point beach, AL uses almost all of the available
covariates. This is due to the covariate selection criterion we used
(\(\text{AIC}_{\tt c}\)) which is intended to minimize predictive error.
As the amount of data increases, we accumulate enough information to
discern an effect even of covariates that are only slightly correlated
with the response. As our dataset grows, then, we should expect more
covariates to be selected for an AL model, and Point has far more
observations than the other sites.

\section{Discussion}\label{discussion}

The GBM-CV, GBM-OOB, and AL methods showed the best results by both
PRESS and AUROC, under LOO and LOYO cross validation. Though GBM-CV was
a bit more accurate than GBM-OOB in all the settings, the small
improvement in accuracy may not outweigh the large additional cost in
computational time to fit the model. However, the additional
computational cost is incurred only once when the model is estimated -
given a new observation of beach data, both the GBM-CV and GBM-OOB
models produce predictions nigh-instantaneously. Where predictive
accuracy is the most important consideration and no difficulty is
anticipated in acquiring the data, it is hard to argue against using a
GBM-type model.

The predictive performance of the AL models was somewhat worse than that
of the GBM models, but by including a covariate selection step, the AL
models reduce the number of covariates that must be measured in order to
make daily predictions. A model that requires fewer covariates is less
expensive and more robust (as the probability of encountering some
missing data increases with the number of required covariates). This is
particularly important for manually-collected covariates because
collecting data by hand takes more time and is more costly than
accessing publically available data from a web service. Across all of
the sites, the ratio of manually-collected to automatically collected
covariates in the AL models seems to mirror the ratio among all
available covariates, indicating that neither the manually- nor
automatically-collected covariates are systematically more important to
predicting the bacterial concentration. Some covariates tended to appear
at every site in the AL models (and other models that include a
covariate selection step). The manually-collected covariates that were
consistently selected for the models were the (log) turbidity in the
beach water, and wave height at the beach.

Where minimizing the number of covariates is important, the selection
criterion used here (\(\text{AIC}_{\tt c}\)) may not be ideal. In that
case it may be advantageous to use a criterion that is more parsimonious
about including covariates in the model, such as the Bayesian
information criterion (BIC), for which the penalty term
\(2df + 2df(df+1)/(n-df-1)\) in (\ref{eq:AICc-continuous}) or
(\ref{eq:AICc-binary}) would be replaced by \(n \times df\), which grows
with the sample size \(n\). The BIC does not exhibit the property of the
AIC (or \(\text{AIC}_{\tt c}\)) where more covariates are included in
the model as the number of observations increases. However, the BIC is
derived from the standpoint of identifying the most probable model,
rather than minimizing the predictive error. It is therefore likely that
an AL model using the BIC for covariate selection will have slightly
worse predictive performance than one using the \(\text{AIC}_{\tt c}\).

Another advantage of the AL over GBM-type models is interpretability. As
a linear regression technique, fitting an AL model means generating a
set of coefficients, which can be interpreted as the marginal effect of
a change in the corresponding covariate. On the other hand, GBM produces
black-box models that typically make more accurate predictions but are
difficult to intrepret. One common way to interpret a random forest
model (such as from the GBM algorithm) is to observe the proportion of
splits in the underlying trees that involve a particular covariate. The
split proportion is a measurement of that covariate's importance to the
model but gives no indication of how that covariate affects the
bacterial concentraion.

All statistical methods and the comparison for this study were carried
out in the R statistical software environment \citep{R-2014}. Scripts and
details of the how the modeling methods were implemented are in the
online supplement. Often times, beach management practitioners are not
very familiar with statistical analysis and rely on more accessible
software to help guide them through development of models for
recreational water quality predictions. For this purpose, the Virtual
Beach software was developed \citep{VB3-2013}. The GA, PLS, and GBM
algorithms used in this study are based on simplified versions of the
comparable algorithms in Virtual Beach version 3.0. An implementation of
AL is also an anticipated addition to Virtual Beach.

\section{Acknowledgments}\label{acknowledgments}

The predictive models for this study were generated on facilities and
software (HTCondor) provided by the University of Wisconsin-Madison's
Center for High Throughput Computing.

\section{Figure Captions}

\subsection{Figure 1}
Map showing the location of the seven Wisconsin beaches for which models were analyzed in this work.

\subsection{Figure 2}
Mean ranking of the methods across all seven sites by area
under the receiver operating characteristic curve (AUROC). The error bars are
90\% confidence intervals computed by the bootstrap. At left are the
AUROC rankings from the leave-one-year-out cross validation (a), at
right are the AUROC rankings from the leave-one-out cross validation
(b).

\subsection{Figure 3}
{Mean ranking of the methods by predictive error sum of squares
(PRESS) across all sites (higher is better). The error bars are 90\%
confidence intervals computed by the bootstrap. At left are the PRESS
rankings from the leave-one-year-out cross validation (a), at right are
the PRESS rankings from the leave-one-out cross validation (b).

\subsection{Figure 4}
At each site, the number of predictions from AL, GBM-OOB, and
GBM-CV that fell into four categories, from left: true negatives, false
positives, true positive, and false negatives.

\subsection{Figure 5}
At each site, the mean number of covariates that were selected
for the AL model, and the total number of covariates, all of which were
used in the gradient boosting machine with an out-of-bag estimate of the
optimal tree count (GBM-OOB) models. For both AL and GBM-OOB, the
covariate counts are broken down by whether the covariate values were
collected automatically from web services or manually at the beach.

\section{References}\label{references}
\bibliographystyle{chicago}
\begin{thebibliography}{}

\bibitem[\protect\citeauthoryear{Akaike}{Akaike}{1973}]{Akaike-1973}
Akaike, H. (1973).
\newblock Information theory and an extension of the maximum likelihood
  principle.
\newblock In {\em Second international symposium on information theory}, pp.\
  267--281. Akademinai Kiado.

\bibitem[\protect\citeauthoryear{Brandt, Schwab, Croley, Belestky, and
  Whitman}{Brandt et~al.}{2006}]{Brandt:2006gj}
Brandt, S., D.~Schwab, T.~Croley, D.~Belestky, and R.~Whitman (2006).
\newblock {\em Ecosystem Forecasting: Integrating Science to Reduce the Risks
  to Human Health}.
\newblock American Geophysical Union.

\bibitem[\protect\citeauthoryear{Brooks, Fienen, and Corsi}{Brooks
  et~al.}{2013}]{Brooks-Fienen-Corsi-2013}
Brooks, W.~R., M.~N. Fienen, and S.~R. Corsi (2013).
\newblock Partial least squares for efficient models of fecal indicator
  bacteria on great lakes beaches.
\newblock {\em Journal of Environmental Management\/}~{\em 114}, 470--475.

\bibitem[\protect\citeauthoryear{Cabelli}{Cabelli}{1983}]{Cabelli:1983od}
Cabelli, V.~J. (1983).
\newblock Health effects criteria for marine recreational waters.
\newblock Tech report EPA-600/1-80-031, {United States Environmental Protection
  Agency Office of Research and Development}.

\bibitem[\protect\citeauthoryear{Cabelli, Dufour, Levin, McCabe, Haberman, and
  Jensen}{Cabelli et~al.}{1979}]{Cabelli:1979lb}
Cabelli, V.~J., A.~P. Dufour, M.~A. Levin, L.~J. McCabe, P.~W. Haberman, and
  L.~D. Jensen (1979).
\newblock Relationship of microbial indicators to health effects at marine
  bathing beaches.
\newblock {\em American Journal of Public Health\/}~{\em 69\/}(7), 690--696.

\bibitem[\protect\citeauthoryear{Chun and Keles}{Chun and
  Keles}{2010}]{Chun-Keles-2010}
Chun, H. and S.~Keles (2010).
\newblock Sparse partial least squares regression for simultaneous dimension
  reduction and variable selection.
\newblock {\em Journal of the Royal Statistical Society: Series B (Statistical
  Methodology)\/}~{\em 72\/}(1), 3--25.

\bibitem[\protect\citeauthoryear{Cyterski, Brooks, Galvin, Wolfe, Carvin,
  Roddick, Fienen, and Corsi}{Cyterski et~al.}{2013}]{VB3-2013}
Cyterski, M., W.~Brooks, M.~Galvin, K.~Wolfe, R.~Carvin, T.~Roddick, M.~Fienen,
  and S.~Corsi (2013).
\newblock {\em Virtual Beach 3: User's Guide}.
\newblock United States Environmental Protection Agency.

\bibitem[\protect\citeauthoryear{Cyterski, Zhang, White, Molina, Wolfe, Parmar,
  and Zepp}{Cyterski
  et~al.}{2012}]{Cyterski-Zhang-White-Molina-Wolfe-Parmar-Zepp-2012}
Cyterski, M., S.~Zhang, E.~White, M.~Molina, K.~Wolfe, R.~Parmar, and R.~Zepp
  (2012).
\newblock Temporal synchronization analysis for improving regression modeling
  of fecal indicator bacteria levels.
\newblock {\em Water, Air \& Soil Pollution\/}~{\em 223}, 4841--4851.

\bibitem[\protect\citeauthoryear{de~Brauwere, Ouattara, and
  Servais}{de~Brauwere et~al.}{2014}]{deBrauwere-Koffi-Servais-2014}
de~Brauwere, A., N.~K. Ouattara, and P.~Servais (2014).
\newblock Modeling fecal indicator bacteria concentrations in natural surface
  waters: a review.
\newblock {\em Critical Reviews in Environmental Science and Technology\/}~{\em
  44\/}(21), 2380--2453.

\bibitem[\protect\citeauthoryear{Dufour}{Dufour}{1984}]{Dufour:1984yn}
Dufour, A.~P. (1984).
\newblock Health effects criteria for fresh recreational waters.
\newblock Tech report EPA-600/1-84-004, {United States Environmental Protection
  Agency Office of Research and Development}.

\bibitem[\protect\citeauthoryear{Fleisher, Fleming, Solo-Gabriele, Kish,
  Sinigalliano, Plano, Elmir, Wang, Withum, Shibata, Gidley, Abdelzaher, He,
  Ortega, Zhu, Wright, Hollenbeck, and Backer}{Fleisher
  et~al.}{2010}]{Fleisher:2010xo}
Fleisher, J.~M., L.~E. Fleming, H.~M. Solo-Gabriele, J.~K. Kish, C.~D.
  Sinigalliano, L.~Plano, S.~M. Elmir, J.~D. Wang, K.~Withum, T.~Shibata, M.~L.
  Gidley, A.~Abdelzaher, G.~He, C.~Ortega, X.~Zhu, M.~Wright, J.~Hollenbeck,
  and L.~C. Backer (2010).
\newblock The {BEACHES} study: health effects and exposures from non-point source
  microbial contaminants in subtropical recreational marine waters.
\newblock {\em International Journal of Epidemiology\/}~{\em 39\/}(5),
  1291--1298.

\bibitem[\protect\citeauthoryear{Fogel}{Fogel}{1998}]{Fogel-1998}
Fogel, D.~B. (1998).
\newblock {\em Evolutionary computation: the fossil record}.
\newblock Wiley-IEEE Press.

\bibitem[\protect\citeauthoryear{Francy, Brady, Carvin, Corsi, Fuller,
  Harrison, Hayhurst, Lant, Nevers, Terrio, and Zimmerman}{Francy
  et~al.}{2013}]{Francy-et-al-2013}
Francy, D.~S., A.~M.~G. Brady, R.~B. Carvin, S.~R. Corsi, L.~M. Fuller, J.~H.
  Harrison, B.~A. Hayhurst, J.~Lant, M.~B. Nevers, P.~J. Terrio, and T.~M.
  Zimmerman (2013).
\newblock Developing and implementing the use of predictive models for
  estimating water quality at {G}reat {L}akes beaches.
\newblock Scientific Investigations Report 2013-5166, United States Geological
  Survey.

\bibitem[\protect\citeauthoryear{Francy and Darner}{Francy and
  Darner}{2007}]{Francy:2007yv}
Francy, D.~S. and R.~A. Darner (2007).
\newblock Nowcasting beach advisories at {Ohio Lake Erie Beaches}.
\newblock Open File Report 2007-1427, United States Geological Survey.

\bibitem[\protect\citeauthoryear{Frick, Ge, and Zepp}{Frick
  et~al.}{2008}]{Frick:2008jo}
Frick, W.~E., Z.~Ge, and R.~G. Zepp (2008).
\newblock Nowcasting and forecasting concentrations of biological contaminants
  at beaches: A feasibility and case study.
\newblock {\em Environmental Science \& Technology\/}~{\em 42\/}(13),
  4818--4824.

\bibitem[\protect\citeauthoryear{Friedman}{Friedman}{2001}]{Friedman-2001}
Friedman, J. (2001).
\newblock Greedy function approximation: a gradient boosting machine.
\newblock {\em Annals of Statistics\/}, 1189--1232.

\bibitem[\protect\citeauthoryear{Ge and Frick}{Ge and Frick}{2007}]{Ge:2007ou}
Ge, Z. and W.~E. Frick (2007).
\newblock Some statistical issues related to multiple linear regression
  modeling of beach bacteria concentrations.
\newblock {\em Environmental Research\/}~{\em 103\/}(3), 358--364.

\bibitem[\protect\citeauthoryear{Hanley and McNeil}{Hanley and
  McNeil}{1982}]{Hanley-McNeil-1982}
Hanley, J.~A. and B.~J. McNeil (1982).
\newblock The meaning and use of the area under a receiver operating
  characteristic ({ROC}) curve.
\newblock {\em Radiology\/}~{\em 148\/}(1), 29--36.

\bibitem[\protect\citeauthoryear{He and He}{He and He}{2008}]{He:2008jx}
He, L.-M.~L. and Z.-L. He (2008).
\newblock Water quality prediction of marine recreational beaches receiving
  watershed baseflow and stormwater runoff in southern {C}alifornia, {USA}.
\newblock {\em Water research\/}~{\em 42\/}(10), 2563--2573.

\bibitem[\protect\citeauthoryear{Hosmer~Jr and Lemeshow}{Hosmer~Jr and
  Lemeshow}{2004}]{Hosmer-Lemeshow-2004}
Hosmer~Jr, D.~W. and S.~Lemeshow (2004).
\newblock {\em Applied logistic regression}.
\newblock John Wiley \& Sons.

\bibitem[\protect\citeauthoryear{Hou, Rabinovici, and Boehm}{Hou
  et~al.}{2006}]{Hou:2006nf}
Hou, D., S.~J.~M. Rabinovici, and A.~B. Boehm (2006).
\newblock Enterococci predictions from partial least squares regression models
  in conjunction with a single-sample standard improve the efficacy of beach
  management advisories.
\newblock {\em Environmental Science \& Technology\/}~{\em 40\/}(6),
  1737--1743.

\bibitem[\protect\citeauthoryear{Hurvich, Simonoff, and Tsai}{Hurvich
  et~al.}{1998}]{Hurvich-Simonoff-Tsai-1998}
Hurvich, C.~M., J.~S. Simonoff, and C.-L. Tsai (1998).
\newblock Smoothing parameter selection in nonparametric regression using an
  improved {A}kaike information criterion.
\newblock {\em Journal of the Royal Statistical Society: Series B
  (Methodology)\/}~{\em 60\/}(2), 271--293.

\bibitem[\protect\citeauthoryear{Jin and Englande~Jr.}{Jin and
  Englande~Jr.}{2006}]{Jin:2006tr}
Jin, G. and A.~J. Englande~Jr. (2006).
\newblock Prediction of swimmability in a brackish water body.
\newblock {\em Management of Environmental Quality\/}~{\em 17\/}(2), 197--208.

\bibitem[\protect\citeauthoryear{Jones, Liu, and Dorovitch}{Jones
  et~al.}{2012}]{Jones-Liu-Dorovitch-2012}
Jones, R.~M., L.~Liu, and S.~Dorovitch (2012).
\newblock Hydrometeorological variables predict fecal indicator bacteria
  densities in freshwater: data-driven methods for variable selection.
\newblock {\em Environmental Monitoring and Assessment\/}~{\em 185\/}(3),
  2355--2366.

\bibitem[\protect\citeauthoryear{Kashefipour, Lin, and Falconer}{Kashefipour
  et~al.}{2005}]{Kashefipour-Lin-Falconer-2005}
Kashefipour, S.~M., B.~Lin, and R.~A. Falconer (2005).
\newblock Neural networks for predicting seawater bacterial levels.
\newblock {\em Proceedings of the Institution of Civil Engineers-Water
  Management\/}~{\em 158\/}(3), 111--118.

\bibitem[\protect\citeauthoryear{National environmental methods index}{National
  environmental methods index}{2013}]{Colilert}
National environmental methods index (2013).
\newblock {\em Colilert Test Kit Procedure}.
\newblock National environmental methods index.

\bibitem[\protect\citeauthoryear{{National Oceanic and Atmospheric
  Administration}}{{National Oceanic and Atmospheric
  Administration}}{2012}]{NCRFC}
{National Oceanic and Atmospheric Administration} (2012).
\newblock North {C}entral {R}iver {F}orecasting {C}enter.

\bibitem[\protect\citeauthoryear{Nevers and Whitman}{Nevers and
  Whitman}{2005}]{Nevers:2005ln}
Nevers, M.~B. and R.~L. Whitman (2005).
\newblock Nowcast modeling of \emph{{E}scherichia coli} concentrations at multiple
  urban beaches of southern {Lake Michigan}.
\newblock {\em Water research\/}~{\em 39\/}(20), 5250--5260.

\bibitem[\protect\citeauthoryear{Olyphant and Whitman}{Olyphant and
  Whitman}{2004}]{Olyphant:2004yq}
Olyphant, G.~A. and R.~L. Whitman (2004).
\newblock Elements of a predictive model for determining beach closures on a
  real time basis: the case of 63rd {S}treet {B}each {C}hicago.
\newblock {\em Environmental Monitoring and Assessment\/}~{\em 98}, 175--190.

\bibitem[\protect\citeauthoryear{Parkhurst, Brenner, Dufour, and
  Wymer}{Parkhurst et~al.}{2005}]{Parkhurst:2005zf}
Parkhurst, D.~F., K.~P. Brenner, A.~P. Dufour, and L.~J. Wymer (2005).
\newblock Indicator bacteria at five swimming beaches - analysis using random
  forests.
\newblock {\em Water Research\/}~{\em 39\/}(7), 1354--1360.

\bibitem[\protect\citeauthoryear{{R Core Team}}{{R Core Team}}{2014}]{R-2014}
{R Core Team} (2014).
\newblock {\em {R: A Language and Environment for Statistical Computing}}.
\newblock Vienna, Austria: R Foundation for Statistical Computing.

\bibitem[\protect\citeauthoryear{Schwab and Bedford}{Schwab and
  Bedford}{1999}]{Schwab-Bedford-1999}
Schwab, D.~J. and K.~W. Bedford (1999).
\newblock The {Great Lakes} forecasting system.
\newblock {\em Coastal and Estuarine Studies\/}, 157--174.

\bibitem[\protect\citeauthoryear{Stidson, Gray, and McPhail}{Stidson
  et~al.}{2012}]{Stidson-2012}
Stidson, R.~T., C.~A. Gray, and C.~D. McPhail (2012).
\newblock Development and use of modelling techniques for real-time bathing
  water quality predictions.
\newblock {\em Water and Environment Journal\/}~{\em 26\/}(1), 7--18.

\bibitem[\protect\citeauthoryear{Thoe, Gold, Griesbach, Grimmer, Taggart, and
  Boehm}{Thoe et~al.}{2014}]{Thoe-Gold-Griesbach-Grimmer-Taggart-Boehm-2014}
Thoe, W., M.~Gold, A.~Griesbach, M.~Grimmer, M.~L. Taggart, and A.~B. Boehm
  (2014).
\newblock Predicting water quality at {S}anta {M}onica {B}each: evaluation of
  five different models for public notification of unsafe swimming confditions.
\newblock {\em Water Research\/}~{\em 67}, 105--117.

\bibitem[\protect\citeauthoryear{Tibshirani}{Tibshirani}{1996}]{Tibshirani-1996}
Tibshirani, R. (1996).
\newblock Regression shrinkage and selection via the lasso.
\newblock {\em Journal of the Royal Statistical Society, Series B
  (Methodological)\/}, 267--288.

\bibitem[\protect\citeauthoryear{{United States Environmental Protection
  Agency}}{{USEPA}}{1986}]{USEPA:ecs}
{United States Environmental Protection Agency} (1986).
\newblock Ambient water quality criteria for bacteria.
\newblock Technical Report EPA-440-5-84-00.

\bibitem[\protect\citeauthoryear{{United States Environmental Protection
  Agency}}{{USEPA}}{2007}]{USEPA:2007lj}
{United States Environmental Protection Agency} (2007).
\newblock Critical path science plan for the development of new or revised
  recreational water quality criteria.
\newblock Technical Report EPA-823-R-08-002.

\bibitem[\protect\citeauthoryear{{United States Environmental Protection
  Agency}}{{USEPA}}{2008}]{USEPA:2008ib}
{United States Environmental Protection Agency} (2008).
\newblock {G}reat {L}akes beach sanitary survey user's manual.
\newblock Technical Report EPA-823-B-06-001.

\bibitem[\protect\citeauthoryear{{United States Environmental Protection
  Agency}}{{USEPA}}{2012}]{USEPA-2012}
{United States Environmental Protection Agency} (2012).
\newblock Recreational water quality criteria.
\newblock Technical Report EPA-820-F-12-058.

\bibitem[\protect\citeauthoryear{{United States Geological Survey}}{{USGS}}{2014a}]{EnDDaT-2014}
{United States Geological Survey} (2014a).
\newblock Environmental {D}ata {D}iscovery and {T}ransformation.

\bibitem[\protect\citeauthoryear{{United States Geological Survey}}{{USGS}}{2014b}]{NWIS}
{United States Geological Survey} (2014b).
\newblock The {N}ational {W}ater {I}nformation {S}ystem.

\bibitem[\protect\citeauthoryear{Wade, Calderon, Brenner, Sams, Beach,
  Haugland, Wymer, and Dufour}{Wade et~al.}{2008}]{Wade:2008yi}
Wade, T.~J., R.~L. Calderon, K.~P. Brenner, E.~Sams, M.~Beach, R.~Haugland,
  L.~Wymer, and A.~P. Dufour (2008).
\newblock High sensitivity of children to swimming-associated gastrointestinal
  illness: results using a rapid assay of recreational water quality.
\newblock {\em Epidemiology\/}~{\em 19\/}(3), 375--383.

\bibitem[\protect\citeauthoryear{Wade, Calderon, Sams, Beach, Brenner,
  Williams, and Dufour}{Wade et~al.}{2006}]{Wade:2006qc}
Wade, T.~J., R.~L. Calderon, E.~Sams, M.~Beach, K.~P. Brenner, A.~H. Williams,
  and A.~P. Dufour (2006).
\newblock Rapidly measured indicators of recreational water quality are
  predictive of swimming-associated gastrointestinal illness.
\newblock {\em Environmental Health Perspectives\/}~{\em 114\/}(1), 24--28.

\bibitem[\protect\citeauthoryear{Waschbusch, Corsi, Sorsa, Walker, Standridge,
  and Schnieder}{Waschbusch et~al.}{2004}]{Waschbusch:2004bd}
Waschbusch, R., S.~Corsi, K.~Sorsa, J.~Walker, J.~Standridge, and T.~Schnieder
  (2004).
\newblock Final report for the {EMPACT} project: data collection and modeling
  of enteric pathogens, fecal indicators and real-time environmental data at
  {Madison, Wisconsin} recreational beaches for timely public access to water
  quality information.
\newblock Technical Report R-82933901-0, Madison Beach EMPACT Team.

\bibitem[\protect\citeauthoryear{Whitman and Nevers}{Whitman and
  Nevers}{2004}]{Whitman:2004pc}
Whitman, R.~L. and M.~B. Nevers (2004).
\newblock Escherichia coli sampling reliability at a frequently closed {C}hicago
  beach: Monitoring and management implications.
\newblock {\em Environmental Science \& Technology\/}~{\em 38\/}(16),
  4241--4246.

\bibitem[\protect\citeauthoryear{Whitman and Nevers}{Whitman and
  Nevers}{2008}]{Whitman:2008nb}
Whitman, R.~L. and M.~B. Nevers (2008).
\newblock Summer \emph{{E}. coli} patterns and responses along 23 {C}hicago beaches.
\newblock {\em Environmental Science \& Technology\/}~{\em 42\/}(24),
  9217--9224.

\bibitem[\protect\citeauthoryear{Whitman, Nevers, Korinek, and
  Byappanahalli}{Whitman et~al.}{2004}]{Whitman:2004wv}
Whitman, R.~L., M.~B. Nevers, G.~C. Korinek, and M.~N. Byappanahalli (2004).
\newblock Solar and temporal effects on {E}scherichia coli concentration at a
  {Lake Michigan} swimming beach.
\newblock {\em Applied and Environmental Microbiology\/}~{\em 70\/}(7),
  4276--4285.

\bibitem[\protect\citeauthoryear{{Wisconsin Department of Natural
  Resources}}{{Wisconsin Department of Natural Resources}}{2012}]{WDNR-2012}
{Wisconsin Department of Natural Resources} (2012).
\newblock Wisconsin's {G}reat {L}akes beach monitoring and notification
  program.
\newblock Technical report.

\bibitem[\protect\citeauthoryear{Wold, Sjostrom, and Eriksson}{Wold
  et~al.}{2001}]{Wold-Sjostrum-Eriksson-2001}
Wold, S., M.~Sjostrom, and L.~Eriksson (2001).
\newblock Pls-regression: a basic tool of chemometrics.
\newblock {\em Chemometrics and intelligent laboratory systems\/}~{\em
  58\/}(2), 109--130.

\bibitem[\protect\citeauthoryear{Zou}{Zou}{2006}]{Zou-2006}
Zou, H. (2006).
\newblock The adaptive lasso and its oracle properties.
\newblock {\em Journal of the American Statistical Association\/}~{\em
  101\/}(476), 1418--1429.

\end{thebibliography}

\end{document}

